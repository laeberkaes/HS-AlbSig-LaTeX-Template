\appendix
\addcontentsline{toc}{part}{Appendix}
\part*{\textsf{Appendix}}
\section{\texttt{/etc/clamav/clamd.conf} Beispiel}\label{app:clamd.conf}
\begin{lstlisting}[language=]	
	# Set the  mount point to be scanned. The mount point specified, or the mount
	# point containing the specified directory will be watched. If any directories
	# are specified, this option will preempt (disable and ignore all options
	# related to) the DDD system. This option will result in verdicts only.
	# Note that prevention is explicitly disallowed to prevent common, fatal
	# misconfigurations. (e.g. watching "/" with prevention on and no exclusions
	# made on vital system directories)
	# It can be used multiple times.
	# Default: disabled
	OnAccessMountPath /
	#OnAccessMountPath /home/<user>
	
	# Modifies fanotify blocking behaviour when handling permission events.    
	# If off, fanotify will only notify if the file scanned is a virus,    
	# and not perform any blocking.    
	# Default: no    
	OnAccessPrevention no 
	
	# Toggles extra scanning and notifications when a file or directory is    
	# created or moved.    
	# Requires the  DDD system to kick-off extra scans.    
	# Default: no    
	OnAccessExtraScanning yes
	
	# Execute a command when virus is found. In the command string %v will
	# be replaced with the virus name.                                      
	# Default: no
	#VirusEvent /usr/local/bin/send_sms 123456789 "VIRUS ALERT: %v"                
	VirusEvent /etc/clamav/detected.sh 
	
	# With this option you can whitelist the root UID (0). Processes run under
	# root with be able to access all files without triggering scans or
	# permission denied events.
	# Default: no
	OnAccessExcludeRootUID yes 
	
	# This option allows exclusions via user names when using the on-access
	# scanning client. It can be used multiple times.
	# It has the same potential race condition limitations of the
	# OnAccessExcludeUID option.
	# Default: disabled
	OnAccessExcludeUname clamav
\end{lstlisting}

\newpage
\section{\texttt{/etc/clamav/detected.sh} Beispiel}
\begin{lstlisting}
	#!/bin/bash
	PATH=/usr/bin
	alert="Signature detected: $CLAM_VIRUSEVENT_VIRUSNAME in $CLAM_VIRUSEVENT_FILENAME"
	
	if [[ -z $(command -v systemd-cat) ]]; then
		echo "$(date) - $alert" >> /var/log/clamav/detections.log
	else
		echo "$alert" | /usr/bin/systemd-cat -t clamav -p emerg
	fi
	
	XUSERS=($(who|awk '{print $1$NF}'|sort -u))
	
	for XUSER in $XUSERS; do
		NAME=(${XUSER/(/ })
		DISPLAY=${NAME[1]/)/}
		DBUS_ADDRESS=unix:path=/run/user/$(id -u ${NAME[0]})/bus
		echo "run $NAME - $DISPLAY - $DBUS_ADDRESS -" >> /tmp/testlog 
			/usr/bin/sudo -u ${NAME[0]} DISPLAY=${DISPLAY} DBUS_SESSION_BUS_ADDRESS=${DBUS_ADDRESS} PATH=${PATH} /usr/bin/notify-send -i dialog-warning "clamAV" "$alert"
	done
\end{lstlisting}

\newpage
\section{\ac{BIOS} Passwort setzen}
\begin{figure}[!h]
	\centering
	\begin{subfigure}{.85\textwidth}
		\centering
		\includegraphics[width=\linewidth]{./abbildungen/main}
		\caption{\ac{BIOS} Hauptmenü}
		\label{fig:main}
	\end{subfigure}

\vspace{1.5cm}

	\begin{subfigure}{.85\textwidth}
		\centering
		\includegraphics[width=\linewidth]{./abbildungen/sec}
		\caption{Security Untermenü}
		\label{fig:sec}
	\end{subfigure}
\end{figure}

\begin{figure}[T]
	\ContinuedFloat
	\centering
	\begin{subfigure}{.85\textwidth}
		\centering
		\includegraphics[width=\linewidth]{./abbildungen/pass}
		\caption{Passwort Einstellung}
		\label{fig:pass}
	\end{subfigure}
\caption{Einrichtung eines \ac{BIOS} Passwortsd}
\label{fig:bios}
\end{figure}