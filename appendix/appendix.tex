\appendix
\addcontentsline{toc}{part}{Appendix}
\part*{\textsf{Appendix}}
% Hier wird dann mit A, B, ... nummeriert.
\section{Codebeispiele}
Hier kommen dann Codestücke rein, die für den Fließtext zu lang oder nicht relevant genug waren.

\begin{lstlisting}[language=c,label=lst:code]
	#include <stdio.h>
	
	const double mwst_satz = 0.19;
	const double artikel_1 = 7.20;
	const double artikel_2 = 1.40;
	const double artikel_3 = 5.60;
	double preis;
	
	void add_mwst(void) {
		printf("\n+\t%f EUR\n", preis * (mwst_satz+1));
	}
	
	int main(void) {
		preis = artikel_1;
		add_mwst();
		preis = artikel_2;
		add_mwst();
		preis = artikel_3;
		add_mwst();
		printf("----------\n");
		preis = artikel_1 + artikel_2 + artikel_3;
		add_mwst();
		printf("==========\n");
		getchar();
		return 0;
	}
\end{lstlisting}

\section{Bilder}
Hier kommen Bilder rein, die zu viel für den Fließtext waren.

\begin{figure}[!h]
	\centering
	\includegraphics[width=.4\textwidth]{Logo2.png}
	\caption{Das ist die Bildunterschrift}
	\label{fig:figref}
\end{figure}