% !TEX root = ../vorlage.tex
% !TeX spellcheck = de_DE
%!TEX root=../vorlage.tex
\section{Kapitel}\label{sec:kapitel}
Wen labels vergeben wurden, kann darauf auch referenziert werden. Code sieht man beispielsweise in \autoref{lst:code} und eine Abbildung in \autoref{fig:figref}

\subsection{Tabelle}
\begin{longtable}{p{.2\textwidth}|p{.7\textwidth}}
	\toprule
	{\large\textsc{Linke Spalte}} & {\large\textsc{Rechte Spalte}} \\
	\midrule[1pt]
	\texttt{Beispiel\autocite{example}} & Das ist die \ac{Abk.} von einem \ac{Bsp.}. Wiederholte Einträge währen kurz. (\ac{Abk.} oder \ac{Bsp.}) \\
	\bottomrule
	\caption{Auflistung zusätzlicher Pakete}
	\label{tab:software}
\end{longtable}

\subsubsection{Unterunterüberschrift}\label{subsubsec:label}
Unterüberschriften sollten nur, wenn unbedingt nötig benutzt werden.
