%!TEX root=../vorlage.tex
\documentclass[a4paper, 11pt]{article}

%%%%%%%%%%%%%%%%%%%%%%%%%%%%%%%%%%%%%%%%%%%%%%%%%%%%%%%%%%%
%%%%%%%%%%%%%%%%%%%%%%%%%ALLGEMEIN%%%%%%%%%%%%%%%%%%%%%%%%%
%%%%%%%%%%%%%%%%%%%%%%%%%%%%%%%%%%%%%%%%%%%%%%%%%%%%%%%%%%%

%\usepackage{polyglossia} % deutsches Sprachpaket
%\setdefaultlanguage{german}
\usepackage[ngerman]{babel}
\usepackage{fontenc}
\usepackage[utf8]{inputenc}
\usepackage{csquotes}
\usepackage{fullpage}
\usepackage[top=2.5cm, bottom=2.5cm, left=2.75cm, right=2.75cm]{geometry} % gibt die Seitenränder an

%%%%%Microtype Verbesserungen%%%%%%
\usepackage[activate={true,nocompatibility},final,tracking=true,kerning=true,spacing=true,factor=1100,stretch=10,shrink=10]{microtype} % microtype Verbesserungen

\SetExtraKerning[unit=space]
    {encoding={*}, family={bch}, series={*}, size={footnotesize,small,normalsize}}
    {\textendash={400,400}, % en-dash, add more space around it
     "28={ ,150}, % left bracket, add space from right
     "29={150, }, % right bracket, add space from left
     \textquotedblleft={ ,150}, % left quotation mark, space from right
     \textquotedblright={150, }} % right quotation mark, space from left
\SetExtraKerning[unit=space]
  {encoding={*}, family={qhv}, series={b}, size={large,Large}}
	{1={-200,-200}, \textendash={400,400}}
\SetTracking{encoding={*}, shape=sc}{40}

\usepackage{graphicx} %Abbildungen einfügen
\usepackage[printonlyused]{acronym} %Abkürzungsverzeichnis
%\usepackage{acro}
\usepackage[hyphens]{url}		% Für Zeilenumbruch bei URLS
%\setcounter{biburllcpenalty}{7000} %Erlaubt Zeilenumbrüche bei ULRs bei Biblatex
%\Urlmuskip=0mu plus 1mu

\usepackage{chngcntr}		%Ändert Bildnummerierung von durchgehende Nummerierung
\counterwithin{figure}{section}		%Kapitel.Bildnummer
\counterwithin{table}{section}		%Kapitel.Tabellennummer

\usepackage{verbatim}
\usepackage{enumitem} %damit man Aufzählungen auch ohne Aufzählungszeichen verwenden kann
\usepackage[colorlinks=false,pdfborder={0 0 0},bookmarksnumbered]{hyperref} %für Hyperlinks und Querverweise
\usepackage{ftnxtra} %Für Zitate in Captions
\interfootnotelinepenalty=10000 %footnotes werden nicht gebrochen

% Fußnoten
\usepackage[hang]{footmisc}
\setlength{\skip\footins}{.75cm}
\setlength{\footnotesep}{0.4cm}

% Hurenkinder und Schusterjungen verhindern
\clubpenalty10000
\widowpenalty10000
\displaywidowpenalty=10000

%Verschiedene Fonts
%\usepackage{charter} %Ohne Mathfont
%\usepackage[bitstream-charter]{mathdesign} % mit Mathfont
\usepackage{kpfonts}%  for math
%\usepackage{fourier}
%\usepackage{libertine}%  serif and sans serif
%\usepackage{palatino}
%\usepackage[scaled=.8]{beramono}%% mono
\usepackage[scaled=.95]{nimbusmono}
%\usepackage[adobe-utopia]{mathdesign}
%\usepackage[scaled=.95]{helvet}
%\usepackage{trajan}

%Header und Footer
\usepackage[headsepline,nouppercase]{scrlayer-scrpage}
\pagestyle{plain}
\automark{section}
\setlength{\headsep}{0.5cm}

%Tausenderstellen Trennen mit einem Leerzeichen
\usepackage{numprint}

%Tabellen mit verschiedener breite
%\usepackage{tabularx}
\usepackage{booktabs}
\usepackage{longtable}
\setlength{\heavyrulewidth}{1.5pt}
\usepackage{makecell}
\usepackage{array}
\newcolumntype{L}[1]{>{\raggedright\let\newline\\\arraybackslash\hspace{0pt}}m{#1}}
\newcolumntype{C}[1]{>{\centering\let\newline\\\arraybackslash\hspace{0pt}}m{#1}}
\newcolumntype{R}[1]{>{\raggedleft\let\newline\\\arraybackslash\hspace{0pt}}m{#1}}
\renewcommand{\arraystretch}{1.5}

%PDF Dateien als ganze Seiten einfügen
\usepackage{pdfpages}

%URL mit Sonderzeichern (\verb|text|) in Fußnoten
% \urldef{\urlA}\url{https://apps.fcc.gov/oetcf/eas/reports/ViewExhibitReport.cfm?mode=Exhibits&RequestTimeout=500&calledFromFrame=N&application_id=SzCBk%2Brpg3ieoz2CJAFlbQ%3D%3D&fcc_id=NDD9530401309}

%Bilder in float
\usepackage{float}

%Rahmen für Bilder
\fboxrule=.75pt

% Platz zwischen Überschrift und Text
\usepackage[sf,explicit,bf]{titlesec}
\titlespacing\section{0pt}{12pt plus 4pt minus 2pt}{4pt plus 2pt minus 2pt}
\titlespacing\subsection{0pt}{12pt plus 4pt minus 2pt}{0pt plus 2pt minus 2pt}
\titlespacing\paragraph{0pt}{4pt plus 4pt minus 2pt}{12pt plus 4pt minus 4pt}

% Umlauf für Bilder
%\usepackage{wrapfig}

% Punkte zwischen Abschnitten und Seitenzahlen
%\usepackage{tocloft}
%\renewcommand{\cftsecleader}{\cftdotfill{\cftdotsep}}
%\renewcommand{\cftsubsecleader}{}

% Makros
\newcommand{\paragraphs}[1]{\paragraph*{#1}{\hspace{-1em}}~~--~~}
\newcommand{\subparagraphit}[1]{\subparagraph{\textit{#1\hspace{1em}--}}}
\newcommand{\itembf}[1]{\item{\textbf{#1}}}

% Keine bold Subparagraphen
%\makeatletter
%\renewcommand\subparagraph{\@startsection{subparagraph}{4}{\z@}%
%	{3.25ex \@plus1ex \@minus.2ex}%
%	{-1em}%
%	{\sffamily}}
%\makeatother

% Farbe für inline Listing
\definecolor{light-gray}{gray}{0.95}
%\newcommand{\code}[1]{\colorbox{light-gray}{\lstinline{#1}}}
% \newcommand{\code}[1]{\texttt{#1}}

%Für Grafiken
\usepackage{tikz}
\usetikzlibrary{shapes.multipart,arrows}
\tikzstyle{partition} = [rectangle, minimum width=3cm, minimum height=1cm, text centered, draw=black, fill=blue!30]
\tikzstyle{line} = [thick, -, >=stealth]
\tikzstyle{arrow} = [thick, ->, >=stealth]

%Worttrennungen
%\input{./hyphenation/hyphenation}

%Spacing
\usepackage{setspace}

% Captions sind kleiner geschrieben und in Kapitälchen
\usepackage[font=small,labelfont=sc]{caption}
\usepackage[font=small]{subcaption}

% Acro lange Form in italic
\renewcommand*{\acffont}[1]{\textit{#1}}
\renewcommand*{\acfsfont}[1]{\textnormal{\small#1}}

% Zitate
%\usepackage{epigraph}
\usepackage[font=itshape]{quoting}

% ToDos
\usepackage[obeyFinal,
            textsize    = footnotesize,
            figwidth    = 0.99\linewidth,
            ngerman,
            colorinlistoftodos,
            ]{todonotes}

\usepackage{kantlipsum}
