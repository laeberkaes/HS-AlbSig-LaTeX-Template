%!TEX root=../HA.tex
\section{Résumé und Ausblick}
In der vorliegenden Arbeit wurden Punkte vorgestellt, an denen Veränderungen vorgenommen werden können, um ein Arch Linux Desktopsystem abzusichern. Dabei wurden Punkte inkludiert, die noch vor der Installation zu beachten sind. Weiter wurden Konfigurationen aufgezeigt, welche die Sicherheit erhöhen. Zum Schluss sind Programme und deren Konfiguration vorgestellt worden, mit welchen (Arch) Linux Systeme vor bestimmten Bedrohungen geschützt werden können. Die Darstellungen sollen keines Falls abschließend sein. Die Absicherung eines Systems kann nahezu beliebig ausgeweitet werden. So sollen die Quellen dieser Arbeit zur weiteren Vertiefung einladen, um das System nach eigenem Belieben weiter oder weniger weit abzusichern.

In Zukunft können weitere Arbeiten verfasst werden, die entweder andere Themengebiete oder Distributionen beschreiben oder tiefer in die vorgestellten Mechanismen eingehen. Auch könnten Profile für bestimmte Gebiete erstellt werden {\small(Was muss bei der Einrichtung von Linux Systemen in der öffentlichen Verwaltung in puncto Sicherheit beachtet werden? Was muss bei der Verwendung in Schulen beachtet werden? $\dots$)}. Der Verfasser geht davon aus, dass sich die aktuelle Entwicklung fortsetzt und Linux Desktopsysteme weiter Verbreitung finden werden, weswegen das Thema Absicherung dieser Systeme an Bedeutung wachsen wird.

Auch wenn bei längerer Beschäftigung der Absicherung eines Linux Systems immer weitere Möglichkeiten auftauchen, die umgesetzt werden können, sollte immer eine Abwägung zwischen Sicherheit und Nutzbarkeit des Systems getroffen werden.\cite{EssaysPsychologySecurity2008}

\begin{quoting}
	\centering
	It is possible to tighten the security so much as to make your system unusable. The trick is to secure it without overdoing it.\cite{SecurityArchWiki}
\end{quoting}